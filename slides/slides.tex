% This text is proprietary.
% It's a part of presentation made by myself.
% It may not used commercial.
% The noncommercial use such as private and study is free
% Nov. 2006
% Author: Sascha Frank 
% University Freiburg 
% www.informatik.uni-freiburg.de/~frank/
%
% additional usepackage{beamerthemeshadow} is used
%  
%  \beamersetuncovermixins{\opaqueness<1>{25}}{\opaqueness<2->{15}}
%  with this the elements which were coming soon were only hinted

\documentclass{beamer}
%\usepackage{beamerthemeshadow}
%\usetheme{Warsaw}
\usetheme{Amsterdam}
\usepackage[catalan]{babel}
\usepackage[utf8]{inputenc}
\usepackage{url}
\begin{document}
\title{Algorismes genètics aplicats a ciències de la vida}
\author{Raimon Grau Cuscó}
\date{\today}

\AtBeginSection[]
{
\begin{frame}
    \frametitle{Índex}
    \tableofcontents[currentsection]
\end{frame}
}

\frame{\titlepage} 

\frame{\frametitle{index}\tableofcontents} 

\section{Motivació} % (fold)
\label{sec:Motivacio}
\begin{itemize}
	\item hola
	\item nanana ananan \ref{sec:Chiron}
	\item adeu
\end{itemize}
% section Motivació (end)


\section{Introducció} % (fold)

\begin{frame}
	\begin{columns}[c]
		\column{0.5\textwidth}
		\framebox{\includegraphics[height=0.9\textheight,width=0.9\textwidth]{images/StrangeLoop.jpg}}
		\column{0.5\textwidth}
		\begin{itemize}
			\item hola
				\pause
			\item matat
		\end{itemize}
	\end{columns}
\end{frame}



\section{Algorismes evolutius} % (fold)
\label{sec:Algorismes evolutius}

\begin{frame}
	\frametitle{Funcionament bàsic}
	\begin{columns}[c]
		\column{0.5\textwidth}
		\framebox{\includegraphics[height=0.4\textheight,width=0.9\textwidth]{images/ga.png}}
		\column{0.5\textwidth}
		\begin{itemize}
			\item 
				\pause
			\item matat
		\end{itemize}
	\end{columns}
\end{frame}

\begin{frame}
	\frametitle{Individu}
	\begin{block}{Què és?}
		Representació dels trets rellevants d'una possible solució al problema.
	 \end{block}
	\pause
	\begin{itemize}
		\item Ha de cumplir algunes condicions bàsiques per al problema (llargada, un cert crc,
			etc.) Durant tota la seva vida.
		\item No tots els trets del individu en l'entorn estan determinats pel seu codi genètic.
			(genotip vs fenotip)
	\end{itemize}
\end{frame}

\begin{frame}
	\frametitle{Fitness: Com de bo és un individu?}
	\begin{itemize}
		\item Com de bo un individu?
			\pause
		\item dfas
	\end{itemize}
\end{frame}

\begin{frame}
	\frametitle{Inicialització}
	\begin{columns}[c]
		\column{0.5\textwidth}
		\framebox{\includegraphics[height=0.7\textheight,width=0.9\textwidth]{images/giraffe-crossover.jpg}}
		\column{0.5\textwidth}
		\begin{itemize}
			\item Primera població.
			\item Es generen els individus aleatòriament.
			\item Si hi ha restriccions, s'han de complir.
		\end{itemize}
	\end{columns}
\end{frame}

\begin{frame}
	\frametitle{Avaluació}
	\begin{columns}[c]
		\column{0.5\textwidth}
		\framebox{\includegraphics[height=0.7\textheight,width=0.9\textwidth]{images/giraffe-crossover.jpg}}
		\column{0.5\textwidth}
		\begin{itemize}
			\item 
				\pause
			\item matat
		\end{itemize}
	\end{columns}
\end{frame}

\begin{frame}
	\frametitle{Selecció}
	\begin{columns}[c]
		\column{0.5\textwidth}
		\framebox{\includegraphics[height=0.7\textheight,width=0.9\textwidth]{images/giraffe-crossover.jpg}}
		\column{0.5\textwidth}
		\begin{itemize}
			\item 
				\pause
			\item matat
		\end{itemize}
	\end{columns}
\end{frame}

\begin{frame}
	\frametitle{Creuament}
	\begin{columns}[c]
		\column{0.5\textwidth}
		\framebox{\includegraphics[height=0.7\textheight,width=0.9\textwidth]{images/giraffe-crossover.jpg}}
		\column{0.5\textwidth}
		\begin{itemize}
			\item 
				\pause
			\item matat
		\end{itemize}
	\end{columns}
\end{frame}

\begin{frame}
	\frametitle{Mutació}
	\begin{columns}[c]
		\column{0.5\textwidth}
		\framebox{\includegraphics[height=0.7\textheight,width=0.9\textwidth]{images/giraffe-crossover.jpg}}
		\column{0.5\textwidth}
		\begin{itemize}
			\item 
				\pause
			\item matat
		\end{itemize}
	\end{columns}
\end{frame}

\begin{frame}
	\frametitle{Creuament }
	\begin{columns}[c]
		\column{0.5\textwidth}
		\framebox{\includegraphics[height=0.7\textheight,width=0.9\textwidth]{images/giraffe-crossover.jpg}}
		\column{0.5\textwidth}
		\begin{itemize}
			\item 
				\pause
			\item matat
		\end{itemize}
	\end{columns}
\end{frame}
% section Algorismes evolutius (end)

\section{Pholus} % (fold)
\label{sec:Pholus}

\begin{frame}
	\frametitle{Creuament }
	\begin{itemize}
		\item 
	\end{itemize}
\end{frame}
% section Pholus (end)

\section{Chiron} % (fold)
\label{sec:Chiron}
% section Chiron (end)
\begin{frame}
	\frametitle{Química combinatòria}
	\begin{itemize}
		\item Esquelet que presenta certa activitat
		\item Hi ha punts on podem unir-hi seqüències d'atoms
		\item Aquestes seqüències estan ja definides quines poden haver-hi en cada punt d'unió
	\end{itemize}
	\pause
	\begin{block}{Objectiu}
		Volem trobar la combinació de substituients que maximitzi la bondat del esquelet
	\end{block}
\end{frame}

\begin{frame}
\end{frame}
\begin{frame}
	\frametitle{API}
	\begin{itemize}
		\item \textbf{newExp} (\$user, \$name, \$numAleles, \$popSize, \$cached , \$bounds)
		\item \textbf{removeExp} (\$id\_exp)
		\item \textbf{listExp} (\$userName)
		\item \textbf{setFitness} (\$id\_ind, \$f)
		\item \textbf{nextIteration} (\$id\_exp)
		\item \textbf{listIndividualsByExp} (\$id\_exp)
		\item \textbf{wipe}
	\end{itemize}
\end{frame}

    %listIndividualsByExp
    %setFitness
    %nextIteration
    %wipe
    %sum
    %print2
    %gep
    %bye
    %hi
    %number
    %number2
    %execu2
    %execu

\section{GEP} % (fold)
\label{sec:GEP}
% section GEP (end)

\section{Conclusions} % (fold)
\label{sec:Conclusions}
\begin{frame}
	i tu ke tal
\end{frame}

\subsubsection{hola matat} % (fold)
\label{ssub:hola matat}
\begin{frame}
	jo bé
\end{frame}

% subsubsection hola matat (end)

% section Conclusions (end)


%\section{Objetivos y motivación} 
%\frame{\frametitle{Problemática} 
%Los hdd crecen y crecen, y nosotros no somos capaces de organizar toda la info
%\pause
%\begin{itemize}
%\item  Crear un programa de fácil interface, que permita moverme por mi hdd.
%\item  Funciona en X, para cualquier wm.
%\item  Tiene las mínimas dependencias posibles.
%\item  Tiene que ser 'intuitivo' para los usuarios (DWIM)
%\item  Tiene que ser extensible (a mas o menos nivel)
%\end{itemize}
%}

%\frame{\frametitle{¿Qué aplicaciones existen?}
%\begin{itemize}
%\item  Beagle \\
%Daemons!!
%\pause
%\item  Catfish \\
%- Mala interficie y limitado en configuracion. \\
%+ puede usar varios backends
%\pause
%\item  Google Desktop \\
%Linux?
%\pause
%\end{itemize}
%}

%\frame{\frametitle{Requerimientos}
%\begin{itemize}
%\item  Permite buscar y abrir archivos rápidamente
%\item  Ejecuta aplicaciones
%\item  Busca archivos remotamente
%\item  Permite gestionar playlists
%\item  Ayuda a navegar por las ventanas abiertas
%\item  No requiere utilizar ningún módulo ni ningun gestor de ventanas para funcionar (no todas las funcionalidades estaran activas)
%\item  Mínima heurística para facilitar su uso
%\end{itemize}
%}

%\section{Aproximación} 
%\subsection{Primera Aproximación}
%\frame{\frametitle{Buscando Ayudita}
%\begin{block}{find}
%find \~/ -name ``.*\$1.*''
%\end{block}
%\pause
%\begin{block}{ find2perl }
%find2perl [paths] [predicates] \textbar perl
%\end{block}
%}

%\begin{frame}[fragile]
%\frametitle{Buscando Ayudita(II)}
%\begin{block}{File::Find}
%\begin{verbatim}
%use File::Find; 
%find(\&wanted, @directories_to_search); 
%sub wanted { ... }
%\end{verbatim}
%\end{block}
%\pause
%\begin{block}{find2perl}
%\begin{verbatim}
%* File::Find::Rule
%# find all the .pm files in @INC
%my @files = File::Find::Rule->file()
%->name( '*.pm')
%->in(@INC);
%\end{verbatim}
%\end{block}
%\end{frame}

%\begin{frame}[fragile]
%\frametitle{¿Buscamos cada vez?}
%\begin{block} {File::Locate}
%\begin{verbatim}
%use File::Locate;
%print join "\n", locate "^/usr", 
%-rex => 1, "/usr/var/locatedb";
%\end{verbatim}
%\end{block}
%\pause
%\begin{block} {File::Locate::Harder}
%\begin{verbatim}
%my $flh = File::Locate::Harder->new( db => $db_file );
%my $results_aref = $flh->locate( $search_pattern,
%{ case_insensitive => 1,
%regexp => 1, });
%\end{verbatim}
%\end{block}
%\end{frame}

%\begin{frame}[fragile]
%\frametitle{¿Qué escojemos?}
%\pause
%\begin{center}
%\Huge{NINGUNA} \\
%\pause
%\Large{Portabilidad (GetOpt::Long; Pod::Usage)} \\
%\begin{verbatim}
%slocate -d $dbname $caseflag -r 
%".*$pattern[^/]*\.$ft\$"
%\end{verbatim}
%\end{center}
%\end{frame}


%\subsection{Interfeisss}
%\frame{
%\frametitle{Interfícies}
%\begin{block}{Tk}
%Complicado \\
%No standard
%\end{block}
%\pause
%\begin{block}{Zenity}
%zenity --info --text=``JAPH'' \\
%zenity --list --print-column 2 --hide-column 2 label out
%\end{block}
%\pause
%\begin{block}{ratmen}
%\$ENV\{RATPOISON\} -c `` echo YAPH ''  \\
%ratmen etiqueta1 comando1 etiqueta2 comando2
%\end{block}
%}

%\section{Código} 
%\begin{frame}
%\frametitle{Algoritmo}
%\begin{itemize}
%\item Checkear archivo de configuración
%\item Preguntar por una extensión/comando/orden
%\begin{itemize}
%\item si es extensión, pedir un patrón.
%\item si es una aplicación, ejecutarla
%\item si es una orden, ejecutarla. (gen, ff, vi)
%\end{itemize}
%\item Si es una extensión nueva pedimos una aplicación con la que abrir esos archivos.    
%\end{itemize}
%\begin{center}
%Usamos prefijos para distinguir entre .tex (archivo) y tex(aplicación)
%\end{center}
%\end{frame}

%\begin{frame}
%\frametitle{Prefijos y heurística}
%\begin{itemize}
%\item  .extensión
%\item -ejecutable
%\item ,ventana
%\pause
%\item Si no ponemos ningún prefijo, y entramos una extensión conocida, rf lo interpreta como EXTENSION.
%\item Si no es conocida y es un comando, se ejecuta el comando
%\item Si no es conocida y no es ningun comando, se agrega como nueva extensión
%\end{itemize}
%\end{frame}

%\begin{frame}[fragile]
%\frametitle{Recordar Extensiones}
%\begin{verbatim}
%lastft: pdf
%pdf xpf
%java gvim -f
%chm kchmviewer
%mp3 mocp -c -p -a
%\end{verbatim}
%\pause
%\begin{exampleblock}{Mejora}
%Pasarlo a codigo perl, y organizarlo como un hash, pudiendo
%hacer referencias a otras entradas: \\
%\$ext{ebook}=qw/pdf chm/; \$ext{chm}='kchmviewer';... 
%\end{exampleblock}
%\end{frame}

%\begin{frame}
%\frametitle{Flags}
%El usuario tiene que poder usar el programa sin tocar ni una linea de código.
%\begin{itemize}
%\item --all
%\item --case
%\item --conffile
%\item --loop (not implemented)
%\item --man
%\item --menuexe
%\item --remote
%\item --rootdir
%\end{itemize}
%\pause
%\begin{center}
%\large{GetOpt::Long y posibilidad de meter flags ``en vivo''}
%\end{center}
%\end {frame}

%\begin{frame}[fragile]
%\frametitle{GRID::Machine}
%El usuario puede definir una lista de servers donde tiene clave ssh SIN
%passphrase.
%\begin{verbatim}
%if ( $remote && eval{require GRID::Machine}){
%foreach (@remoteHosts){
%$m = GRID::Machine->new( host => $_ );
%for my $file (searchRemote($_,qq{slocate ... }, $m)){
%$file =~ m/.*\/(.*)$/;
%push @FileList, {path => $file ,name => "$1",
%machine => $m, IP => $_, app => ''};
%}
%}
%}
%\end{verbatim}
%\end{frame}


%\begin{frame}[fragile]
%\frametitle{GRID::Machine (II)}
%\begin{small}
%\begin{verbatim}
%$FileList[$res]{'machine'}->get(
%[$FileList[$res]{'path'}], '/tmp/');
%$^T = time();
%system($application. ' ' .quotemeta("/tmp". 
%"/$FileList[$res]{'name'}"));
%my $age = -M quotemeta("/tmp"."/$FileList[$res]{'name'}");
%$FileList[$res]{'path'} =~ s/(.*\/)[^\/]*/$1/mig;
%if ($^T == $age) {
%$FileList[$res]{'machine'}->put(["/tmp
%/$FileList[$res]{'name'}"],
%qq{$FileList[$res]{'path'}});
%}
%\end{verbatim}
%\end{small}
%\end{frame}

%\begin{frame}[fragile]
%\frametitle{Identificar paths con ``tilde''}
%\begin{verbatim}
%--rootdir=~/
%--rootdir=~foo
%--rootdir=~foo/bar

%$ext_app_file =~ s{ ^ ~ ( [^/]* ) }
%{ $1
%? (getpwnam($1))[7]
%: ( $ENV{HOME} || $ENV{LOGDIR}
%|| (getpwuid($>))[7]
%)
%}ex;
%\end{verbatim}
%\end{frame}

%\section{Un entorno de trabajo completo}
%\subsection{plugins}
%\begin{frame}[fragile]
%\frametitle{Define con WWW::Dictionary}
%\begin{verbatim}
%WWW::Dictionary;
%my $dictionary = WWW::Dictionary->new();
%my $meaning = $dictionary->meaning( $word );
%\end{verbatim}
%\pause
%\begin{verbatim}
%use WWW::Dictionary;
%my $dictionary = WWW::Dictionary->new( join('', @ARGV) );
%@_ = split ('\n' , $dictionary->get_meaning);
%print grep {/Spanish/} @_ , "\n";
%\end{verbatim}
%\end{frame}

%\begin{frame}[fragile]
%\frametitle{Syntax con WWW::Mechanize}
%HTML::Strip - Perl extension for stripping HTML markup from text
%\begin{verbatim}
%use WWW::Mechanize;
%use HTML::Strip;

%$w1 = WWW::Mechanize->new;
%$w1->get("http://www.google.es/search....");
%$html = HTML::Strip->new;
%$html->parse($w1->content);
%$html =~ m/de aproximadamente ([\d.]+) paginas/;
%\end{verbatim}
%\end{frame}


%\begin{frame}
%\frametitle{Trabajo Futuro}
%\begin{itemize}
%\item Refactorizar código
%\item Utilizar sqlite para localizar archivos.
%\item hacer 'plugins' para buscar streamings de música
%\item --loop flag
%\item --tmpcmd flag
%\end{itemize}
%\end{frame}

%\begin{frame}
%\frametitle{Bibliografia}	
%\begin{itemize}
%\item \url{http://software.twotoasts.de/?page=catfish}
%\item \url{http://beagle-project.org/Main\_Page}
%\item \url{www.cpan.org}
%\item O'Reilly Perl  Bookshelf (Beginning Perl,  Programming  Perl, Perl Cookbook...) 
%\item \url{http://use.perl.org/~Alias/journal/36415}
%\end{itemize}
%\end{frame}

%\begin{frame}
%Gracias por su atención
%\huge{\centering{http://code.google.com/p/ratfinder/}}
%\end{frame}


\end{document}
