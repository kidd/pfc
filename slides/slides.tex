% This text is proprietary.
% It's a part of presentation made by myself.
% It may not used commercial.
% The noncommercial use such as private and study is free
% Nov. 2006
% Author: Sascha Frank 
% University Freiburg 
% www.informatik.uni-freiburg.de/~frank/
%
% additional usepackage{beamerthemeshadow} is used
%  
%  \beamersetuncovermixins{\opaqueness<1>{25}}{\opaqueness<2->{15}}
%  with this the elements which were coming soon were only hinted

\documentclass{beamer}
%\usepackage{beamerthemeshadow}
%\usetheme{Warsaw}
\usetheme{Amsterdam}
\usepackage[catalan]{babel}
\usepackage[utf8]{inputenc}
\usepackage{url}
\begin{document}
\title{Algorismes genètics aplicats a ciències de la vida}
\author{Raimon Grau Cuscó}
\date{\today}

\AtBeginSection[]
{
\begin{frame}
    \frametitle{Índex}
    \tableofcontents[currentsection]
\end{frame}
}

\frame{\titlepage} 

%\frame{\frametitle{index}\tableofcontents} 

\section{Motivació} % (fold)
\label{sec:Motivacio}
\begin{itemize}
	\item hola
	\item nanana ananan \ref{sec:Chiron}
	\item adeu
\end{itemize}
% section Motivació (end)


\section{Introducció} % (fold)

\begin{frame}
	\begin{columns}[c]
		\column{0.5\textwidth}
		\framebox{\includegraphics[height=0.9\textheight,width=0.9\textwidth]{images/StrangeLoop.jpg}}
		\column{0.5\textwidth}
		\begin{itemize}
			\item hola
				\pause
			\item matat
		\end{itemize}
	\end{columns}
\end{frame}



\section{Algorismes evolutius} % (fold)
\label{sec:Algorismes evolutius}

\begin{frame}
	\frametitle{Funcionament bàsic}
	\begin{columns}[c]
		\column{0.5\textwidth}
		\framebox{\includegraphics[height=0.4\textheight,width=0.9\textwidth]{images/ga.png}}
		\column{0.5\textwidth}
		\begin{itemize}
			\item Inicialització
			\item Avaluació
			\item Selecció
			\item Creuament
			\item Mutació
		\end{itemize}
	\end{columns}
\end{frame}

\begin{frame}
	\frametitle{Individu}
	\begin{block}{Què és?}
		Representació dels trets rellevants d'una possible solució al problema.
	 \end{block}
	\pause
	\begin{itemize}
		\item Ha de cumplir algunes condicions bàsiques per al problema (llargada, un cert crc,
			etc.) Durant tota la seva vida.
		\item No tots els trets del individu en l'entorn estan determinats pel seu codi genètic.
			(genotip vs fenotip)
	\end{itemize}
\end{frame}

\begin{frame}
	\frametitle{Inicialització}
	\begin{columns}[c]
		\column{0.5\textwidth}
		\framebox{\includegraphics[height=0.7\textheight,width=0.9\textwidth]{images/giraffe-crossover.jpg}}
		\column{0.5\textwidth}
		\begin{itemize}
			\item Primera població.
			\item Es generen els individus aleatòriament.
			\item Si hi ha restriccions, s'han de complir.
		\end{itemize}
	\end{columns}
\end{frame}

\begin{frame}
	\frametitle{Fitness: Com de bo és un individu? (Avaluació)} 
	\begin{itemize}
		\item És el propi problema.  Fa el paper de l'entorn en el que està
		l'individu.
		\item Cada individu l'enfrontem a la funció de fitness.
		\item Retorna un valor (numèric) que ens permet comparar la bondat dels
		individus entre ells
	\end{itemize}
	\pause
	\begin{alertblock}{Compte!}
		Aquesta funció s'executa moltes vegades s'ha d'anar amb compte amb el
		cost
	\end{alertblock}
\end{frame}

\begin{frame}
	\frametitle{Selecció}
		\begin{block}{Els escollits}
			Quins individus es reproduiràn, per passar el seu codi genètic a la següent
			generació?
		\end{block}
		\pause
		\begin{itemize}
			\item Accedeixen a reproduir-se un percentatge molt alt de la
			població.
			\item Diferents tipus de mecanismes de selecció.
			\item Un o altre mecanisme pot provocar diferències molt grans en
			els resultats del algorisme.
			\item Es el responsable de la pressió evolutiva i la convergència
			prematura.
		\end{itemize}
		%\begin{exampleblock}
		%\end{exampleblock}
\end{frame}

%XXX chicha sobre seleccio

\begin{frame}
	\frametitle{Creuament}
	\begin{block}
		Mecanismes pels cuals creem descendència a partir de 2 individus
		existents.
	\end{block}
		\begin{itemize}
			\item Varia en funció del problema. %\verbatim{1100110011  001100110011}
			\item S'ha de ``pensar'' en el tipus de problema, i decidir
			\item Hi ha creuaments específics per algun tipus de problema (TSP).
		\end{itemize}
\end{frame}

%epistàcia

\begin{frame}
	\frametitle{Mutació}
		\begin{itemize}
			\item Una vegada fet el creuament, un percentatge molt petit de la nova
			població pateix mutacions en algun dels seus gens.
			\item Les mutacions provoquen diferències dràstiques respecte el
			cromosoma no mutat.
			\item Va en contra de la convergència de la població.
		\end{itemize}
\end{frame}


\begin{frame}
\frametitle{Estat de l'art}
\begin{itemize}
\item Des que Holland publica sobre algorismes genètics, s'han utilitzat cada
cop en més àmbits
\item La gran potència de càlcul actual permet la seva utilització per tot tipus
de problemes
\pause
\item Als inicis s'utilitzava lisp.
\item Actualment existeixen llibreries en gairebé tots els llenguatges.
\item En els últims 10 anys, s'han trobat maneres de paralelitzar els algorismes
genètics.
\end{itemize}
\end{frame}

%\begin{frame}
	%\frametitle{Creuament}
	%\begin{columns}[c]
		%\column{0.5\textwidth}
		%\framebox{\includegraphics[height=0.7\textheight,width=0.9\textwidth]{images/giraffe-crossover.jpg}}
		%\column{0.5\textwidth}
		%\begin{itemize}
			%\item 
				%\pause
			%\item matat
		%\end{itemize}
	%\end{columns}
%\end{frame}
%% section Algorismes evolutius (end)

\section{Pholus} % (fold)
\label{sec:Pholus}

\begin{frame}
	\frametitle{Introducció}
	Per detectar similituds entre molècules 
\end{frame}

\begin{frame}
\frametitle{Aproximació}
\end{frame}

\begin{frame}
\frametitle{Funcionament general}
\end{frame}

\begin{frame}
\frametitle{Operadors}
\end{frame}

\begin{frame}
\frametitle{Implementació}
\end{frame}

\begin{frame}
	\frametitle{Creuament}
	\begin{itemize}
		\item fdsa
	\end{itemize}
\end{frame}
% section Pholus (end)

\section{Chiron} % (fold)
\label{sec:Chiron}
% section Chiron (end)
\begin{frame}
	\frametitle{Química combinatòria}
	\begin{itemize}
		\item Esquelet que presenta certa activitat
		\item Hi ha punts on podem unir-hi seqüències d'atoms
		\item Aquestes seqüències estan ja definides quines poden haver-hi en cada punt d'unió
	\end{itemize}
	\pause
	\begin{block}{Objectiu}
		Volem trobar la combinació de substituients que maximitzi la bondat del esquelet
	\end{block}
\end{frame}

\begin{frame}
	\frametitle{Aproximació}
	Construir un programa que permeti trobar la combinació de substituients que
	maximitza la seva activitat.

	El procés ha de ser avaluat sintetitzant cada molècula.

	Utilitzem algorismes evolutius per a optimitzar el número d'avaluacions, ja
	que és un procés molt car.
\end{frame}

\begin{frame}
	\frametitle{Abstracció}
	\begin{itemize}
	\item Aprofitem creuaments i mutacions, però no tenim funció d'avaluació.
	\pause
	\item Podem abstraure el problema i convertir Chiron en un framework
	d'algorismes genètics.
	\end{itemize}
\end{frame}

%explicar chiron en general
\begin{frame}
\frametitle{Funcionament General}
\begin{itemize}
\item Introducció de les característiques del problema (numero de substituients,
i possibles substituients en cada posició)
\item Generació de població inicial
\item Introducció dels fitness de cada individu
\item següent iteració\ldots
\end{itemize}
\pause
L'usuari pot decidir si l'experiment ha de recordar els fitness dels elements ja
evaluats, i no mostrar els individus repetits.
\end{frame}

\begin{frame}
\frametitle{Operadors utilitzats}

 % theme=Berlin;caption=Operadors Chiron
 % Operador & Valor
 % Inicialització & aleatòria
 % Selecció & 0.2
 % Creuament & 0.4
 % Mutació & 0.7

\begin{table}
\centering
\begin{tabular}{|l|l|}
\hline
\multicolumn{1}{|c|}{\textbf{Operador }} & \multicolumn{1}{c|}{\textbf{ Valor}} \\
\hline
\hline
Inicialització & aleatòria \\
Selecció       & 0.2        \\
Creuament       & 0.4        \\
Mutació        & 0.7        \\
\hline
\end{tabular}
\caption{Operadors Chiron}
\end{table}
\end{frame}

\begin{frame}
\frametitle{Problemes d'inicialització}
En el nostre problema real, hi ha molt pocs individus que tinguin un fitness !=
0.
\pause
\begin{alertblock}{Alerta!}
Si en la primera generació tots els individus tenen el mateix fitness, la segona
generació (creada a partir de la primera) tindrà bàsicament elements de la
primera generació, provocant convergència prematura.
\end{alertblock}
\pause
\begin{exampleblock}{Solució}
En la primera generació, ens assegurem que no tots tinguin igual fitness.  Si és
així, creem una població nova aleatòria, per abarcar tot l'espai de búsqueda
\end{exampleblock}
\end{frame}

\begin{frame}
\frametitle{Tecnologies utilitzades}
\begin{tabular}[h!]{|c|c|}
AE & eodev (c++) \\
BBDD & Mysql + DBIX::Class (Perl) \\
WebService & SOAP \\
Enllaç $AE\rightarrow BBDD\rightarrow WS$ & Perl + Template Toolkit\\
Client SOAP & PHP , Perl , C++ \\
\end{tabular}
\end{frame}

\begin{frame}
	\frametitle{API}
	\begin{itemize}
		\item \textbf{newExp} (\$user, \$name, \$numAleles, \$popSize, \$cached , \$bounds)
		\item \textbf{removeExp} (\$idExp)
		\item \textbf{listExp} (\$userName)
		\item \textbf{setFitness} (\$idInd, \$f)
		\item \textbf{nextIteration} (\$idExp)
		\item \textbf{listIndividualsByExp} (\$idExp)
		\item \textbf{wipe}
	\end{itemize}
\end{frame}

\begin{frame}
\frametitle{Resultats}
Chiron s'ha testejat en 3 problemes reals, amb un espai de cerca de més de 15000
possibilitats, i en un XXX\% ha trobat la millor combinació explorant només un
XXX de les possibilitats.
\end{frame}

\section{GEP} % (fold)
\label{sec:GEP}
% section GEP (end)

\section{Conclusions} % (fold)
\label{sec:Conclusions}
\begin{frame}
	i tu ke tal
\end{frame}

\subsubsection{hola matat} % (fold)
\label{ssub:hola matat}
\begin{frame}
	jo bé
\end{frame}

% subsubsection hola matat (end)

% section Conclusions (end)

\end{document}
