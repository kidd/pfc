
%        File: pholus.tex
%     Created: Wed Jul 08 02:00 PM 2009 C
% Last Change: Wed Jul 08 02:00 PM 2009 C
%
\documentclass[a4paper]{article}
\usepackage[spanish]{babel}
\usepackage[utf8]{inputenc}
\begin{document}

\section{Algoritmes Evolutius}\label{sec:ae}

L'aplicació de regles evolutives en problemes de computació va començar en els
anys %XXX 
amb Goldberg

\begin{itemize}
\item Goldberg
\item Koza
\item Ferreira
\item 
\item 
\item 

\end{itemize}

\subsection{Problemes a atacar}\label{sub:problemes a atacar}

El tipus de problemes més adequats per a solucionar amb Algorismes evolutius,
són desde problemes d'optimització, problemes que no sabem a què responen, i
problemes on evolucionen els propis programes.  Aquest últim tipus de problemes,
també són coneguts més específicament com algorismes genètics.  En aquest
projecte, un dels casos a tractar s'ha solucionat utilitzant una modificació
d'algoritme genètic

\end{document}
