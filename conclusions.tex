\chapter{Conclusions} % (fold)
\label{cha:Conclusions}

En la realització d'aquest projecte de final de carrera, s'ha desenvolupat un
conjunt d'eines que permeten millorar la efectivitat real dels processos de
descobriment de nous fàrmacs aplicant algoritmes evolutius amb èxit.
Aquests programes, són eines que no funcionen aïllades, sinó que milloren
processos intermedis que amb les eines disponibles fins al moment, no es feien
tan eficientment, o bé simplement, no existien eines per a tractar aquests
problemes.

Del conjunt d'experiències adquirides en aquest projecte, podem treure diverses
conclusions  Segueixen les tres que resumeixen millor els valors apresos de la
realització d'aquest PFC.

\begin{itemize}
	\item Les eines d'intel·ligència artificial, poden aportar millores reals a
		les metodologies existents avui en dia per a la recerca de nous fàrmacs.
	\item Concretament, els algorismes evolutius han demostrat un cop més que poden ser molt eficaços i competitius en grans espais de cerca, entorns amb difícil avaluació de les potencials solucions i problemes complexos i epistàtics, com els tres problemes on s'ha treballat.
	\item Els programes que creem, han de tenir una interfície adequada, no
		només cap als usuaris finals, sinó que s'han de fer, en la mesura del
		possible reutilitzables, i accessibles des d'altres programes.
		Tecnologies basades en webservices són molt útils per aquestes tasques.
	\item Hofstadter's Law: It always takes longer than you expect, even when
		you take into account Hofstadter's Law \cite{GEB79}.
\end{itemize}

A nivell d'acceptació comercial els tres projectes \texttt{Pholus, Chiron i GEP}
han corregut una sort diferent, que també és interessant fer notar.

\paragraph{Pholus} % (fold)
\label{par:Pholus}

Pholus, ha estat un èxit rotund, i s'està utilitzant actualment en el procés de
HELIOS (Intelligent Pharma), utilitzant-se per extensió en la majoria dels laboratoris
farmacèutics d'Espanya.  És per això que aquest projecte ha sofert diverses
ampliacions i modificacions, de les que no hem parlat en aquest projecte, però
que ens donen una idea del moviment i la utilització que ha sofert aquest
programa. De fet, Pholus està sent utilitzat en aquests moments en una quinzena de projectes de recerca diferents, en àrees terapèutiques tan diverses com oncologia, cardiovascular, hematologia, antibiòtics, etc.

Tot i que l'algorisme evolutiu que hi ha dins de Pholus és aparentment senzill,
molta complexitat ve donada pel context químic en el que es troba.
% paragraph Pholus (end)

\paragraph{Chiron} % (fold)
\label{par:Chiron}

Per la seva part, Chiron és un projecte en estat embrionari, tot i que s'ha testejat amb èxit en alguns problemes reals, la seva comercialització no ha estat massiva.
Per altra banda, Chiron ha estat pensat per a ser un webservice d'algorismes genètics,
i es pot enllaçar amb altres programes, que aporten a Chiron la funció de
avaluació.  Aquest projecte ha requerit el domini de moltes tecnologies
diferents, i ha estat un molt bon exercici d'integració d'eines
(c++, perl, mysql, SOAP, php).
% paragraph Chiron (end)

\paragraph{GEP} % (fold)
\label{par:GEP}
GEP ha estat el projecte que ha tingut menys ``sort'', tot i ser molt
interessant pel grau de recerca bàsica que comporta.  Aquesta evolució de la programació genètica no ens ha permès solucionar els problemes pels que havíem pensat utilitzar-ho, degut a que les funcions que volíem trobar (periòdiques) afegien
dificultat al problema, i tant la falta de temps, com alternatives que s'han
trobat en l'empresa, han fet que no s'investigués més a fons.  De totes maneres,
hem aconseguit resultats similars als últims articles apareguts en la matèria,
en funcions polinòmiques.  Treballar amb tècniques totalment noves (els primers
articles daten de 2001 \cite{ferreira:2001}) ha sigut una experiència molt
estimulant.
% paragraph GEP (end)

Concloem doncs que els algorismes evolutius s'adapten molt bé a la resolució dels tres problemes abordats, que s'han aportat innovacions tècniques importants pel descobriment de nous fàrmacs i amb una gran repercussió industrial, i que el treball en equips de recerca interdisciplinària és estimulant però a l'hora complex en les comunicacions.

% chapter Conclusions (end)

%\section{Programari utilitzat i repositoris} % (fold)
%\label{sec:Programari utilitzat}

%Per a la realització d'aquesta memòria, s'ha utilitzat \latex, vim, emacs, R i
%the gimp.  Tot el software utilitzat és software lliure.  Aquest document es pot
%trobar a \url{http://github.com/kidd/pfc}. 
%% section Programari utilitzat (end)
