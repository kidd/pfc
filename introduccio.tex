\documentclass[a4paper]{article}
\usepackage[catalan]{babel}
\usepackage[utf8]{inputenc}
\begin{document}
La computació evolutiva és un camp recerca dins de la informàtica.  Com el nom
indica, es un camp especial dins la computació, una manera determinada d'atacar
problemes que agafa idees en la evolució natural de les espècies.  No es extrany
que s'hagin intentat portar idees de la naturalesa a camps de la computació, i
menys a camps relacionats amb la inteligència artificial, ja que és ben clar
que, la evolució en aquets cas, ha tingut un fort impacte positiu en les
espècies, a nivell individual, i als ecosistemes, amb multitud d'espècies
convivint i co-evolucionant al llarg del temps.

Un altre cas on s'intenta adaptar els procediments naturals en la computació és,
per exemple, en les ``Xarxes neuronals'', on s'intenta simular les interaccions
de les neurones en un cervell, per a solucionar problemes on hi intervé
l'aprenentatge.

fault-tolerance

En general, els problemes que la evolució intenta solucionar (la evolució no
s'atura, amb el qual mai arriba a un estadi totalment estàtic), els soluciona
mitjançant prova i error.

% TODO : Petita intro a la evolució? 

\section{Breu Història} % (fold)
\label{sec:Breu Historia}

Els principis de la aplicació de les idees de Darwin a la resolució de problemes
daten dels anys quaranta, cuan Alan Turing va proposar ``búsquedes genètiques o
evolutives''.  Al any 1962, Bremermann va executar amb èxit alguns experiments
de ``optimització mitjançant evolució i recombinació''.  Durant els anys
seixanta, es van desenvolupar diferents estudis i implementacions d'aquests
principis en 3 llocs diferents, amb 3 noms lleugerament diferents:


 % theme=Berlin;caption_top=1;caption=Els inicis
 % Nom & Qui & On
 % Fogel, Owens, Walsh & programació evolutiva & EEUU
 % & Algoritme genètic & Holanda
 % Rechenberg, Schwefel & estratègies evolutives & Alemanya

\begin{table}
\centering
\caption{Els inicis}
\begin{tabular}{|l|l|l|}
\hline
\multicolumn{1}{|c|}{\textbf{Nom }} & \multicolumn{1}{c|}{\textbf{ Qui }} & \multicolumn{1}{c|}{\textbf{ On}} \\
\hline
\hline
Fogel, Owens, Walsh  & programació evolutiva  & EEUU     \\
                     & Algoritme genètic      & Holanda  \\
Rechenberg, Schwefel & estratègies evolutives & Alemanya \\
\hline
\end{tabular}
\end{table}

Més endavant, als anys noranta, John Koza va encunyar el terme programació
genètica.  Actualment, totes aquestes variants es conceben dins del que
s'anomenen algorismes evolutius.

% TODO : Ferreira i GEP

Avui dia, i donat que les cerques en problemes NP-Complets son més i més
rellevants, les tècniques englobades dins dels algorismes evolutius estan tenint
cada any més rellevància.  Prova d'això és que ja hi ha vàries conferències
i publicacions dedicades.

\subsection{Inspiració de la Naturalesa} % (fold)
\label{sub:Inspiracio de la Naturalesa}

\subsubsection{Evolució darwiniana} % (fold)
\label{ssub:Evolucio darwiniana}

La teoria de la Evolució de Darwin \cite{Darwin} dóna una explicació de la
diversitat biològica i dels mecanismes de sel·lecció natural.  Aquesta
sel·lecció natural és qui juga el paper principal en la direcció que agafa la
evolució.  Donat un entorn que pot assumir un número limitat de individus, i
l'instint bàsic dels individus a reproduir-se, la sel·lecció esdevé inevitable.
Altrament, la població s'aniria incrementat indefinidament, i l'entorn no podria
suportar-ho.  La sel·lecció natural fa que els individus competeixin pels
recursos existents, i permet només reproduir-se als que estiguin més adaptats al
medi en qüestió.

\subsubsection{Genètica} % (fold)
\label{ssub:Genetica}

% subsubsection Genetica} (end)

% subsubsection Evolucio darwiniana (end)

% subsection Inspiració de la Naturalesa (end)

\section{Introduccio als Algorismes evolutius} % (fold)
\label{sec:Introduccio als AE}

% section Introduccio als AE (end)
% section Breu Història (end)
\end{document}
