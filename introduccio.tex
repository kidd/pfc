\documentclass[a4paper]{article}
\usepackage[catalan]{babel}
\usepackage[utf8]{inputenc}
\begin{document}

La computació evolutiva és un camp recerca dins de la informàtica.  Com el nom
indica, és un camp especial dins la computació, una manera determinada d'atacar
problemes que agafa idees i té la seva inspiració inicial en la evolució natural
de les espècies.  No es extrany que s'hagin intentat portar idees de la
naturalesa a camps de la computació, i menys a camps relacionats amb la
inteligència artificial, ja que és ben clar que, la evolució en aquets cas, ha
tingut un fort impacte positiu en les espècies, a nivell individual, i als
ecosistemes, amb multitud d'espècies convivint i co-evolucionant al llarg del
temps.

Un altre cas on s'intenta adaptar els procediments naturals en la computació és,
per exemple, en les ``Xarxes neuronals'', on s'intenta simular les interaccions
de les neurones en un cervell, per a solucionar problemes on hi intervé
l'aprenentatge.  No parlaré d'aquesta tècnica durant aquest treball, però només
fer notar que la informàtica en general, i la inteligència artificial en
particular, s'ha nodrit molt d'idees i estratègies de la pròpia naturalesa.

Una altra de les particularitats dels algoritmes evolutius, és la seva bona
resposta davant d'irregularitats en les mostres d'on aprèn.  Es diu doncs, que
té una bona tolerància a fallades.  Aquest tret el comparteix amb altres
tècniques basades en l'aprenentatge, com per exemple les xarxes neuronals.
Aquest tret particular, també és una de les característiques que fan dels
algoritmes genètics una molt bona opció davant de problemes dels que tenim dades
experimentals, i no tan sols problemes teòrics, o d'entorns molt controlats.

De la mateixa manera que els algoritmes evolutius són processos estocàstics, que
no sempre donaran la mateixa solució davant d'unes dades d'entrada, tampoc
requereixen unes dades d'entrada perfectament fiables per a donar una solució
més que acceptable al problema.

En general, els problemes que la evolució intenta solucionar (la evolució no
s'atura, amb el qual mai arriba a un estadi totalment estàtic), els soluciona
mitjançant prova i error.  Tant sols conta amb el factor temps, que en cas de la
evolució natural no té final.

% TODO : Petita intro a la evolució? 

\section{Breu Història} % (fold)
\label{sec:Breu Historia}

Els principis de la aplicació de les idees de Darwin a la resolució de problemes
daten dels anys quaranta, cuan Alan Turing va proposar ``búsquedes genètiques o
evolutives''.  Al any 1962, Bremermann va executar amb èxit alguns experiments
de ``optimització mitjançant evolució i recombinació''.  Durant els anys
seixanta, es van desenvolupar diferents estudis i implementacions d'aquests
principis en 3 llocs diferents, amb 3 noms lleugerament diferents:

 % theme=Berlin;caption_top=1;caption=Els inicis
 % Nom & Què & On
 % Fogel, Owens, Walsh & programació evolutiva & EEUU
 % John Henry Holland & Algoritme genètic & 
 % Rechenberg, Schwefel & estratègies evolutives & Alemanya

\begin{table}
\centering
\caption{Els inicis}
\begin{tabular}{|l|l|l|}
\hline
\multicolumn{1}{|c|}{\textbf{Nom }} & \multicolumn{1}{c|}{\textbf{ Què }} & \multicolumn{1}{c|}{\textbf{ On}} \\
\hline
\hline
Fogel, Owens, Walsh  & programació evolutiva  & EEUU     \\
                     & Algoritme genètic      &   \\
Rechenberg, Schwefel & estratègies evolutives & Alemanya \\
\hline
\end{tabular}
\end{table}

Més endavant, als anys noranta, John Koza va encunyar el terme programació
genètica.  Actualment, totes aquestes variants es conceben dins del que
s'anomenen algorismes evolutius.

% TODO : Ferreira i GEP

Avui dia, i donat que les cerques en problemes NP-Complets son més i més
rellevants, les tècniques englobades dins dels algorismes evolutius estan tenint
cada any més rellevància.  Prova d'això és que ja hi ha vàries conferències
i publicacions dedicades.

\subsection{Inspiració de la Naturalesa} % (fold)
\label{sub:Inspiracio de la Naturalesa}

\subsubsection{Evolució darwiniana} % (fold)
\label{ssub:Evolucio darwiniana}

La teoria de la Evolució de Darwin \cite{Darwin} dóna una explicació de la
diversitat biològica i dels mecanismes de sel·lecció natural.  Aquesta
sel·lecció natural és qui juga el paper principal en la direcció que agafa la
evolució.  Donat un entorn que pot assumir un número limitat de individus, i
l'instint bàsic dels individus a reproduir-se, la sel·lecció esdevé inevitable.
Altrament, la població s'aniria incrementat indefinidament, i l'entorn no podria
suportar-ho.  La sel·lecció natural fa que els individus competeixin pels
recursos existents, i permet només reproduir-se als que estiguin més adaptats al
medi en qüestió.  La sel·lecció basada en la competència és una de les dues
pedres angulars del procés evolutiu.  L'altra força primària identificada pels
resultats de Darwin són les variacions en el fenotip dels individus d'una
mateixa població. Els trets del fenotip són aquelles característiques de
comportament i físiques que un individu presenta i que afecten directament a la
seva resposta a determinades situacions i al entorn en general (també als altres
individuus).

Un individu representa una única combinació de trets que són evaluats en conjunt
en un entorn determinat.  Si l'individu és dels que respon millor al medi
(fitness), el seu fenotip serà propagat a la següent generació d'individuus a
través de la reproducció (offspring).  Si la combinació de trets no és de les
millors, l'individu no s'arribarà a reproduir i el seu codi genètic morirà i no
passarà a la següent generació.

\subsubsection{Genètica} % (fold)
\label{ssub:Genetica}

% subsubsection Genetica} (end)

% subsubsection Evolucio darwiniana (end)

% subsection Inspiració de la Naturalesa (end)

\section{Introduccio als Algorismes evolutius} % (fold)
\label{sec:Introduccio als AE}

\url{http://en.wikipedia.org/wiki/Evolutionary\_computation}


% section Introduccio als AE (end)
% section Breu Història (end)

\section{Estat de l'art} % (fold)
\label{sec:Estat de l'art}

Actualment, els algorismes genètics estan essent més i més utilitzats no sols en
el món acadèmic, sinó també en el món de l'empresa.  Per exemplificar això, es
presenten les dades d'un article publicat a 
%XXX 

Les últimes publicacions referents a algorismes genètics tendeixen cap a
optimitzacions de la convergència.  Els algorismes genètics són processos
estocàstics (la durada o nombre d'iteracions fins a aturarse no és un valor
conegut a priori).  Així doncs, el que s'intenta és aconseguir ``guiar''
l'algorisme perquè avanci de pressa cap a un estat convergent (s'aturi), sense
perjudicar això a la eficàcia.Molts dels estudis de tècniques per millorar
l'eficiència dels algorismes genètics estan relacionats amb l'aprofitament de la
paralelització de processos, aprofitant sistemes distribuits, o fins i tot en
cloud.  Més endevant es faran cinc cèntims de tècniques concretes de
paralelització (tot i que no s'han aplicat en aquest projecte en concret).

Els treballs més recents relacionats amb algoritmes evolutius estan relacionats
amb la utilització de entorns de treball (frameworks) i paradigmes de
programació paralela, com poden ser hadoop o MapReduce. XXX Paper 
\url{http://www.xavierllora.net/2009/10/09/scaling-genetic-algorithms-using-mapreduce/}

\subsection{Tecnologia} % (fold)
\label{sub:Tecnologia}

Des dels inicis de la programació 'moderna', el llenguatge per excelència
relacionat amb la Intelligència artificial ha estat \cite{LISP}, creat per John
Macarthy el 1958 donat la gran flexibilitat que oferia (a més, en auqells temps
hi havia fortran com a alternativa, un llenguatge molt més rígid).  Així doncs,
els majors avenços relacionats amb la IA han tingut gairebé sempre les seves
primeres implementacions en lisps o derivats (scheme).  

El problema dels llenguatges tant dinàmics com lisp, és que al treballar sobre
màquina virtual (lisp es va poder compilar a partir del %XXX
) eren massa lents per a la implementació en producció.  Recordem que les
aplicacions que utilitzen algorismes de intelligència artificial requereixen un
gran volum de càlculs.

És per això que els sistemes amb grans necessitats de càlcul per a producció
normalment s'implementen amb llenguatges ``propers al ferro'' com poden ser
C/C++. 

Actualment i donada la gran potència de calcul dels ordenadors actuals, hi ha
empreses que comencen a utilitzar llenguatges interpretats per a realitzar
algorismes genètics  com per exemple \footnote{oblong}.  Llenguatges com haskell
i erlang que han demostrat ser molt ràpids i paralelitzables, no han fet el salt
a la inteligència artificial massivament, ja que la seva puresa (transparència
referencial i fortament tipats) els fa més ferragosos de treballar en problemes
eminentment dinàmics com els algoritmes evolutius  

% subsection Tecnologia (end)

% section Estat de l'art (end)
\end{document}
