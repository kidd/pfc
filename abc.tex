\section{Com Llegir el document} % (fold)
\label{sec:Com Llegir el document}
Aquest projecte de final de carrera, està pensat per ser llegit, o bé d'una
tirada, o bé llegint les seccions per separat, en funció de l'interès del
lector.

En funció del nivell de coneixements prèvis del lector, aquest pot saltar
seccions o bé pel seu nivell o per la temàtica.

El capítol \ref{cha:AG} és una introducció als algorismes evolutius, des del
punt de vista abstracte.  S'explica el funcionament bàsic d'aquests, els
operadors més comuns i un xic de estat de l'art \ref{sec:Estat de l'art}, que
pot interessar i ser útil a persones que no tinguin coneixements particulars
sobre algorismes genètics.  Es fa una introducció també a ``genetic expression
programming'', una disciplina relativament nova, que hem utilitzat per realitzar
la part \ref{cha:GEP} del PFC.

Els següents capítols, corresponen als tres sub-projectes que hem implementat, i
segueixen una estructura bastant similar.  L'apartat de context químic de cada
capítol explica quina és la problemàtica des del punt de vísta de l'àmbit del
projecte.  Aquesta part no és necessaria per entendre les implementacions ni
disssenys, però és on s'explica (simplificadament) els motius químics que han
guiat el desenvolupament, o la elecció d'una tècnica en front d'una altra.

Durant la lectura, el lector o lectora veurà que hi ha paraules en anglès, que
no s'han traduït, o en uns punts s'han traduït i en altres no.  En la major part
dels casos, la primera aparició d'una d'aquestes paraules, va acompanyada de la
paraula original en anglès, i a partir de llavors, pot aparèxier qualsevol de
les dues versions indistintament.  Això ha estat una decisió voluntària, ja que
hi ha paraules eminentment tècniques que aparèixen en molta literatura sense
traduïr i és útil que el lector les conegui, per a futura referència.  Exemples
d'aquestes paraules són \emph{callback, building block, memoize, \ldots}.
% section Com Llegir el document (end)

\section{Programari utilitzat i repositoris} % (fold)
\label{sec:Programari utilitzat}

Per a la realització d'aquesta memòria, s'ha utilitzat \latex, vim, emacs, R i
the gimp.  Tot el software utilitzat és software lliure.  Aquest document es pot
trobar a \url{http://github.com/kidd/pfc}. 
% section Programari utilitzat (end)

