\documentclass[titlepage,a4paper,12pt]{book}

\usepackage[utf8]{inputenc}
\usepackage[catalan]{babel}
\usepackage{graphicx}
\usepackage{verbatim}
\usepackage{marvosym}
\usepackage{amssymb, amsmath} 
\usepackage{listings}
\usepackage{textcomp}
\usepackage[]{color}

%\lstset{
	%language=C++,
	%keywordstyle=\bfseries\ttfamily\color[rgb]{0,0,1},
	%identifierstyle=\ttfamily,
	%commentstyle=\color[rgb]{0.133,0.545,0.133},
	%stringstyle=\ttfamily\color[rgb]{0.627,0.126,0.941},
	%showstringspaces=false,
	%basicstyle=\small,
	%numberstyle=\footnotesize,
	%numbers=left,
	%stepnumber=1,
	%numbersep=10pt,
	%tabsize=2,
	%breaklines=true,
	%prebreak = \raisebox{0ex}[0ex][0ex]{\ensuremath{\hookleftarrow}},
	%breakatwhitespace=false,
	%aboveskip={1.5\baselineskip},
  %columns=fixed,
  %upquote=true,
  %extendedchars=true,
%% frame=single,
%% backgroundcolor=\color{lbcolor},
%}

\begin{document}
\tableofcontents  %%Index

\section{Introducció} % (fold) Objectius?
	\label{sec:Introduccio}
	Tot seguit es presentará un projecte en el que s'ha utilitzat una tècnica
	molt pionera derivada de la programació genètica anomenada ``Genetic
	Expression Programming'' (GEP).


% section Introducció (end)

\section{Context Químic} % (fold)
	\label{sec:Context Quimic}

\section{Procediment informàtic} % (fold)
\label{sec:Procediment informatic}

El diseny i implementació  d'aquest projecte és la que s'ha emportat més temps
proporcionalment, ja que s'ha hagut de fer molta investigació per a ajustar els
paràmetres del algorisme genètic, i per a implementar els diferents operadors.

Ens enfrontem a un problema on el cromosoma, a diferència Pholus i Chiron, no
pot ser tractat com un vector d'elements independents, ja que s'ha de mapejar un
arbre (que representa una fórmula analítica) en un vector, i un gen
\textit{i} té molta repercusió en els seus gens posteriors.

La manera clàssica d'atacar els problemes de programació genètica, és
confeccionant una representació d'un arbre en el cromosoma, en preordre,
inordre, o postordre.  Cadascuna d'aquestes implementacions suposa uns pros i
uns contres, ja sigui en la construcció, la interpretació (per a executar la
funció de fitness s'ha d'evaluar el resultat, i per tant, s'ha de reconstruir
l'arbre), o en els diferents operadors, tant el creuament com la mutació.

Com a primera aproximació, s'ha implementat un arbre en preodre, ja que facilita
molt la evaluació, una de les parts més costoses i que s'ha de fer més cops al
llarg del algoritme.

Imaginem que tenim un arbre de la següent manera:

\begin{verbatim}
+ 2 - 3 2
\end{verbatim}

Aquest arbre, que traduit a inordre és $ 2 + (3 -2) $ , pot ser calculat
construint un evaluador com una màquna de pila, on apilem els elements a mesura
que els anem trobant, i evaluem les operacions una vegada tenim els operands
suficients per a la última operació que hem apilat.

La construcció en preordre deixa les fulles al final dels subarbres, i això és
convenient en la mesura del possible, ja que per als creuaments, augmenta la
possibilitat de crear un arbre vàlid. % XXX REF a creuaments

Un dels problemes que es donen en els algorismes de programació genètica, és la
construcció d'arbres invàlids. Per exemple, suposem que fem un creuament per un
punt de tall entre dos arbres, com si fós un algorisme evolutiu clàssic.

\begin{verbatim}
+ 3 * 3 | - 4 3
- Q + * | 5 3 5
\end{verbatim}

al intercanviar els arbres, un dels 2 arbres que queden (- Q + * - 3 5) no
disposa de prous operands per a realitzar la evaluació.  Això és un cas senzill
de tots els problemes que es poden generar al fer creuaments, i és el cas en el
que utilitzem el creuament més trivial (creuament per un punt). Si utilitzéssim
creuaments més sofisticats, els casos que hem de tractar particularment per
creixen molt de pressa.  Només fent un creuament per 2 punts, es generen
moltíssims més arbres sintacticament incorrectes.

És per això que s'ha adoptat finalment per una aproximació utilitzant una
representació ideada per Ferreira (XXX bib), on per la pròpia construcció del
arbre, podem assegurar que \textbf{sempre} generarem arbres sintàcticament
vàlids.

\subsection{Interfície} % (fold)
\label{sub:Interficie}

L'usuari, en aquest cas encara no està molt definit, ja que segurament, GEP es
farà servir com a complement per a Helios 2.0, però les entrades al programa son
clares:

Es disposa de resultats empírics (o bé per laboratori o bé utilitzant softwares
de simulació) sobre, per exemple, el volum d'una molècula en funció dels angles
dels seus enllaços rotables.

En un cas així, les dades d'entrada són els diferents angles i els volums
obtinguts, i el que es demana a la aplicació, és que a partir d'aquestes dades,
trobi una fórmula analítica que representi (amb la major fiabilitat possible)
aquestes dades, per a poder generalitzar les dades empíriques a una fórmula
tractable matemàticament.

El programa ha d'executar-se d'una tirada, amb el que no tindrà cap component de
interactivitat, ni es necessitarà cap tipus de emmagatzematge de dades
intermitges.

La entrada, doncs, es fa a través d'un paràmetre que ens indica on estan els
resultats que coneixem, juntament amb les dades que ens porten a aquests.

Juntament amb les dades, també s'han d'entrar els diferents paràmetres referents
al algorisme genètic.  Ara per ara no està implementat, però com es veurà en
l'apartat de treballs futurs(XXX), s'ha pensat permetre a l'usuari (se'l
considera un usuari ``expert'', que coneix tant el problema, com té coneixements
d'algorismes genètics) entrar ``building blocks'' addicionals, o
activar/desactivar els que ja venen per defecte.

Això suposa un gran repte a causa de les característiques estàtiques del
llenguatge, però ja s'han estudiat algunes maneres per a poder utilitzar
tècniques que permeten ``dinamitzar'' el llenguatge (utilitzant plugins o
llenguatges de scripting acoblats (embeded) al programa c++.
% subsection Interficie (end)


\subsection{Preparació de les dades} % (fold)
\label{sub:GPreparacio de dades}

La preparació de dades necessària per a aquest projecte no té massa rellevància,
ja que disposem a priori dels conjunts de dades d'entrada i resultats.

El programa simplement agafa les dades d'uns arxius que hem tractat amb uns
petits scripts per a tenir un format realment còmode.

Per a les proves que s'han realitzat per descubrir funcions matemàtiques
conegudes, la mateixa funció s'ha implementat en la funció d'evaluació, i
simplement, s'executa amb les dades d'entrada, per a tenir el resultat a
comparar-lo amb el resultat de la formula que estem evaluant.

% subsection Preparació de les dades (end)

\subsection{Implementació} % (fold)
\label{sub:GImplementacio}

Aquest projecte ha estat el més difícil d'implementar i disenyar, ja que hem
hagut de fer un treball notable d'investigació (els primers referents en GEP
daten de 2002 XXX).

El procés general del algoritme evolutiu és, donada una llargada de cromosoma,
decidida per l'usuari en funció de la complexitat ``intuitiva'' de la formula
que es vol descubrir, i uns paràmetres de probabilitat de creuaments i
mutacions, la aplicació ha de retornar una fórmula analítica que s'apropa (o
clava) la fórmula que estem buscant.
% subsection Impl (end)

(XXX) fer una seccio dedicada a GP i GEP o posar-ho a la implementació?


\section{Algorisme Genètic} % (fold)
\label{sec:GAlgorisme Genetic}

% section Algorisme Genètic (end)

% section Procediment informàtic (end)

    %Introducció
    %Context Químic
    %Procediment informàtic
    %+ Algoritme Genètic
    %+  Individu (Cromosoma)
    %+  Crossover
    %+  Mutacions
    %+ Implementació
    %+  algorisme genètic
    %+  Arxius de configuració
    %+  DBIx::Class
    %+  Webservices i Soap
    %Resultats
    %+ treball futur


i tal i cual
\end{document}
