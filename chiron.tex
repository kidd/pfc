\documentclass[titlepage,a4paper,12pt]{book}

\usepackage[utf8]{inputenc}
\usepackage[catalan]{babel}
\usepackage{graphicx}
\usepackage{marvosym}
\begin{document}
	\section{Introducció} % (fold)
	\label{sec:Introduccio}
	Un altre problema on hem aplicat Algoritmes evolutius és en el descobriment
	de fàrmacs 'de novo'.

	La situació és la següent.  En determinades situacions, es disposa d'un
	esquelet o \textit{scaffold} que se sap que té certa activitat, però no es
	disposa amb seguretat de alguns \textit{radicals} , que són les parts de la
	molècula que, sense canviar la seva estructura general, ni la seva
	activitat, faran que es pugui 'enganxar' millor en el lligand.  Els radicals
	que poden anar en cada posició són coneguts (en la majoria de cassos).

	Un dels problemes que es deriven dels mètodes actuals són els grans costos
	que es deriven d'aquest procés, ja que el que es fa és sintetitzar TOTES les
	possibles variants i combinacions de radicals, i mesurant la energia %XXX (disipada?)
	s'intenta buscar la que minimitza aquesta.

	La idea en aquest programa és construir un algoritme genètic que ens permeti
	arribar a la millor tria de radicals (o alguna de molt bona) sintetitzant
	una petita part del espai de búsqueda.

	Per a evaluar la qualitat de una combinació, no es té més remei que
	sintetitzar la molècula en qüestió en el laboratori i retornar el resultat.
	Així doncs, es tracta d'un algorisme genètic interactiu %XXX ref a paper.
	que ens guiarà les proves a través dels creuaments i les mutacions, trobant
	``relacions'' entre els diferents radicals, i aconseguint resultats bons
	explorant una petita part del espai de búsqueda.

	mataaaaaaaaaaaaaaaaaaaaaaaaaaaaaaaaaaaaaaaaaaaaaaaaaaaaaaaaaaat

	
	
	% section Introducció (end)
\end{document}
