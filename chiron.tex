\documentclass[titlepage,a4paper,12pt]{book}

\usepackage[utf8]{inputenc}
\usepackage[catalan]{babel}
\usepackage{graphicx}
\usepackage{marvosym}
\begin{document}
\section{Introducció} % (fold)
	\label{sec:Introduccio}
	Un altre problema on hem aplicat Algoritmes evolutius és en el descobriment
	de fàrmacs 'de novo'.

	La situació és la següent.  En determinades situacions, es disposa d'un
	esquelet o \textit{scaffold} que se sap que té certa activitat, però no es
	disposa amb seguretat de alguns \textit{radicals} , que són les parts de la
	molècula que, sense canviar la seva estructura general, ni la seva
	activitat, faran que es pugui 'enganxar' millor en el lligand.  Els radicals
	que poden anar en cada posició són coneguts (en la majoria de cassos).

	Un dels problemes que es deriven dels mètodes actuals són els grans costos
	que es deriven d'aquest procés, ja que el que es fa és sintetitzar TOTES les
possibles variants i combinacions de radicals, i mesurant la energia %XXX (disipada?)
	s'intenta buscar la que minimitza aquesta.

	La idea en aquest programa és construir un algoritme genètic que ens permeti
	arribar a la millor tria de radicals (o alguna de molt bona) sintetitzant
	una petita part del espai de búsqueda.

	Per a evaluar la qualitat de una combinació, no es té més remei que
	sintetitzar la molècula en qüestió en el laboratori i retornar el resultat.
	Així doncs, es tracta d'un algorisme genètic interactiu %XXX ref a paper.
	que ens guiarà les proves que hem de fer a través dels creuaments i les
	mutacions, trobant ``relacions'' (epistàcia) entre els diferents radicals, i
	aconseguint resultats bons explorant una petita part del espai de búsqueda.
% section Introducció (end)

\section{Context Químic} % (fold)
	\label{sec:Context Quimic}

	En el proccés de descobriment de fàrmacs (drug discovery), no només és
	necessari trobar un compost que reaccioni favorablement en una molècula
	objectiu, sinó que també ha de reunir certes condicions per tal que un
	principi actiu es pugui convertir en un fàrmac aplicable.  Aquestes
	condicions son tals com:

	\begin{itemize}
	\item No toxicitat
	\item Que es el cos no el rebutgi
	\item Facilitat d'absorció i estabilitat del nou compost (lligand +
			receptor)
	\end{itemize}

% section Context Químic (end)

\section{Algoritme Genètic} % (fold)
	\label{sec:Algoritme Genetic}
% section Algoritme Genetic (end)

\section{Implementació} % (fold)
	\label{sec:Implementacio}
% section Implementació (end)

\section{Resultats} % (fold)
	\label{sec:Resultats}
% section Resultats (end)
	\end{document}
