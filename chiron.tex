\documentclass[titlepage,a4paper,12pt]{book}

\usepackage[utf8]{inputenc}
\usepackage[catalan]{babel}
\usepackage{graphicx}
\usepackage{marvosym}
\begin{document}
\tableofcontents
\section{Introducció} % (fold) Objectius?
	\label{sec:Introduccio}
	Un altre problema on hem aplicat Algoritmes evolutius és en el descobriment
	de fàrmacs 'de novo'.

	La situació és la següent.  En determinades situacions, es disposa d'un
	esquelet o \textit{scaffold} que se sap que té certa activitat, però no es
	disposa amb seguretat de alguns \textit{radicals} , que són les parts de la
	molècula que, sense canviar la seva estructura general, ni la seva
	activitat, faran que es pugui 'enganxar' millor en el lligand.  Els radicals
	que poden anar en cada posició són coneguts (en la majoria de cassos).

	Un dels problemes que tenen els mètodes actuals són els grans costos
	que es deriven d'aquest procés, ja que el que es fa és sintetitzar TOTES les
possibles variants i combinacions de radicals, i mesurant la energia %XXX (disipada?)
	s'intenta buscar la que minimitza aquesta.

	La idea en aquest programa és construir un algoritme genètic que ens permeti
	arribar a la millor tria de radicals (o alguna de molt bona) sintetitzant
	una petita part del espai de búsqueda.

	Per a evaluar la qualitat de una combinació, no es té més remei que
	sintetitzar la molècula en qüestió en el laboratori i retornar el resultat.
	Així doncs, es tracta d'un algorisme genètic interactiu %XXX ref a paper.
	que ens guiarà les proves que hem de fer a través dels creuaments i les
	mutacions, trobant ``relacions'' (epistàcia) entre els diferents radicals, i
	aconseguint resultats bons explorant una petita part del espai de búsqueda.

	Una altra objectiu d'aquest programa, ha estat convertir-lo en un
	``framework'' d'algorismes genètics.  Veient que la funció de fitness és una
	funció que executa l'usuari, i ell es qui puntua cadascuna de les
	combinacions, s'ha intentat anar més enllà i aconseguir una aplicació que
	pugui ser utilitzada no només en aquest context, sinó en cualsevol que ens
	podem trobar en un futur, i que requereixi uns operadors similars.

	Una altra manera d'imaginarse aquest enfoc \texttt{meta} és pensar-ho com si
	separéssim la funció fitness de la resta del procés d'algoritme genètic.
	L'únic que necesita el algoritme genètic és un receptor de dades, que
	donada una entrada, hi doni una sortida.  Si es mira d'aquesta manera, no
	estem fent més que desacoblar la evaluació del procés de creuaments,
	mutacions i reproducció.

	Una vegada estem en aquest punt, la següent evolució lògica és la
	deslocalització també espaial de les 2 parts.  Donat que el procés
	d'evaluació d'una combinació de erres donada pot trigar dies, el coll
	d'ampolla serà clarament aquest, permetent-nos així separar algorisme
	genètic i Fitness a través d'una xarxa, i permetent convertir el sistema en
	un servei web, accessible des d'arreu del món.

	La gestió d'això només ens implica implementar el sistema com a servei web
	(SOAP) i tenir una bona gestió d'usuaris i projectes.

	Tot seguit s'expliquen els detalls d'implementació.

% section Introducció (end)

\section{Context Químic} % (fold)
	\label{sec:Context Quimic}

	En el proccés de descobriment de fàrmacs (drug discovery), no només és
	necessari trobar un compost que reaccioni favorablement en una molècula
	objectiu, sinó que també ha de reunir certes condicions per tal que un
	principi actiu es pugui convertir en un fàrmac aplicable.  Aquestes
	condicions son tals com:

	\begin{itemize}
	\item No toxicitat
	\item Que es el cos no el rebutgi
	\item Facilitat d'absorció i estabilitat del nou compost (lligand +
			receptor)
	\end{itemize}

% section Context Químic (end)

\section{Algoritme Genètic} % (fold)
	\label{sec:Algoritme Genetic}
% section Algoritme Genetic (end)

\section{Implementació} % (fold)
	\label{sec:Implementacio}
% section Implementació (end)

\section{Resultats} % (fold)
	\label{sec:Resultats}
	\subsection{treball futur} % (fold)
	\label{sub:treball futur}
	implementar tècniques de algorisme genètic  interactius
	
	% subsection treball futur (end)
% section Resultats (end)
	\end{document}
