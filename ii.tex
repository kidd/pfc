\section{L’optimització numèrica} Gran part dels problemes a resoldre que es
presenten en els diversos camps de la ciència i la enginyeria es poden abstreure
com a problemes d’optimització numèrica: des de problemes de disseny industrial,
com per exemple l’optimització de peces de motors de combustió, fins al
modelatge de patrons de comportament biològic de models animals
d’experimentació. Sovint, aquests problemes presenten una sèrie de
característiques comunes que els fa complexos i difícils de tractar amb mètodes
d’optimització ‘‘clàssics´´ com poden ser tècniques com en \emph{backtracking},
\emph{branch and bound} o el \emph{simulated annealing}. Aquestes propietats
són, primerament, que la solució està immersa dins d’un enorme espai de cerca
(de vegades infinit i per tant no enumerable), que és costos avaluar la
\emph{bondat} d’una possible solució, que aquesta avaluació no només és costosa
sinó que també és sorollosa i, finalment, que les diferents parts que composen
una solució afecten de manera no lineal i epistàtica la bondat de la solució, és
a dir, que una modificació en una part de la solució pot afectar
significativament  la idoneïtat d’altra part de la solució.

Per tots els motius anteriors, sovint els usuaris es conformen amb no trobar la
solució al problema, sinó a trobar una bona solució. En aquests casos,  les
eines com la computació evolutiva són bons candidats a ser usades com a motor
d’optimització numèrica. La computació evolutiva i en particular els algorismes
genètics són la implementació d’un procés evolutiu darwinià que va ser començat
a ser utilitzat en la dècada dels 70 \cite{H75} i, des d’ençà, els algorismes
evolutius han anat progressant i sent utilitzats cada cop més en diferents
àmbits de l’enginyeria. Fins i tot, i tal i com es veurà en una de les parts del
present PFC, s’han arribat a utilitzar eficaçment per a l’aprenentatge
artificial \cite{G89}.

Justament, els algorismes evolutius estan tenint tant d’èxit en la recerca
industrial perquè aproximen d’una manera correcta les cinc propietats
mencionades anteriorment que confereixen la complexitat als problemes reals.
Potser la propietat dels problemes que menys bé resolen els algorismes evolutius
és l’epistàsia. Per això, a principis dels 2000 van començar a ser desenvolupats
algorismes evolutius d’estimació de distribució, que amb un suficient espai
mostral resolen bé aquest problema \cite{LL02}. En el present PFC no s’ha
treballat amb algorismes d’estimació de distribució però sí amb altres conceptes
moderns que tracten l’epistàsia, com és la meta-genètica \cite{ferreira:2006}, o
en altres paraules, la regulació per vies genètiques de la interacció entre els
gens (o parts de la solució).

\section{El disseny de fàrmacs}

El disseny de nous fàrmacs és una tasca costosa (el preu mitjà de recerca d’un
nou fàrmac és de \$1.000 milions), lenta (el temps mitjà de desenvolupament és de
12 anys) i arriscada (només1 de cada 15 fàrmacs en fases clíniques surt al
mercat). Per això, i degut a que és un assumpte de tanta transcendència social,
calen noves eines que puguin abaratir, accelerar i eliminar riscs en tot el
procés de descobriment de nous fàrmacs.

Abans de les fases clíniques (proves en humans) el procés de disseny un nou fàrmac consta de cinc fases: 1) identificació de \emph{hits}; 2) optimització de \emph{hits}; 3) identificació de \emph{leads}; 4) optimització de \emph{leads}; i 5) la fase preclínica. Les tres parts que composen el present PFC entren en joc en l’etapa 1 (Pholus), 2 (Chiron) i 3 (GEP).

En la etapa d’identificació de \emph{hits} l’objectiu és trobar candidats
moleculars molt primerencs capaços de mostrar un determinat llindar d’activitat
biològica. En aquest context, Pholus és el mòdul de cal·libratge automàtic d’una
altra eina desenvolupada a Intelligent Pharma que té aquest objectiu.

En la segona etapa, un cop identificats els \emph{hits}, aquests s’han
d'optimitzar. Aquesta optimització es fa a laboratori, és a dir, es sintetitzen
sistemàticament moltes variants combinatòries dels \emph{hits} identificats i
s’avaluen biològicament. És el que es coneix com química combinatòria. En aquest
sentit, Chiron és una eina informàtica basada en computació evolutiva que guia
als químics mèdics encarregats de l’optimització molecular en laboratori en la
seva tasca de selecció de variants combinatòries.

Finalment, en la tercera etapa, s’han de realitzar diversos estudis biològics
per identificar quina de les bones molècules és, a més a més, un bon fàrmac
potencial. Per fer-ho, es realitzen una bateria d’estudis molt diversos i
heterogenis com poden ser des d’estudis sobre les propietats toxicològiques dels
\emph{hits} optimitzats fins a les seves capacitats de penetració en membranes
fisiològiques (barrera intestinal, hematoencefàlica, etc.). Per assistir en
aquestes tasques, Intelligent Pharma està en procés de recerca de noves eines
matemàtiques i computacionals que puguin modelar el comportament dels models
biològics usats habitualment en aquesta etapa (toxicologia, permeabilitat,
etc.). El GEP és una part important i complexa dins d’un projecte major que té
aquest objectiu. Concretament, el GEP és la part que permet descriure el
comportament de les propietats fisico-químiques de les molècules en procés
d’estudi mitjançant funcions matemàtiques (en lloc de valor constants, com
clàssicament s’ha fet. Ex: pes molecular, nombre d’anells aromàtics, nombre de
donants de ponts d’hidrogen, etc.). A continuació, i això ja no forma part del
present PFC, aquestes funcions reben un processament funcional per eines
d’aprenentatge artificial com poden ser màquines de suport vectorial.

\section{Els projecte}

Com s’ha mencionat, el present PFC consta de tres grans
parts, totes elles relacionades amb la computació evolutiva aplicada al
descobriment de nous fàrmacs. El PFC s’ha realitzat immers en un entorn
interdisciplinari amb enginyers informàtics, químics computacionals i
matemàtics/estadístics i té una altíssima component de recerca aplicada. 

A més a més de la recerca realitzada, el projecte també ha implicat el
desenvolupament de tres tecnologies que en aquests moments estan sent
utilitzades per la majoria de les empreses farmacèutiques d’Espanya en projectes
reals de recerca de nous medicaments en diverses àrees terapèutiques: oncologia,
malalties neurodegeneratives, cardiovasculars, infeccioses, etc.

La memòria està estructurada en XXX capítols, blablabla
%Paràgraf escrit per Rai on s’explica l’estructura del document.
