\documentclass[titlepage,a4paper,12pt]{book}

\usepackage[utf8]{inputenc}
\usepackage[catalan]{babel}
\usepackage{graphicx}
\usepackage{marvosym}
\usepackage{amssymb, amsmath} 

\begin{document}
\section{Objectius} % (fold)
\label{sec:Objectius}
Aquest projecte anomenat \texttt{Pholus}, ha estat desenvolupat per a equilibrar (ponderar) pesos dels
diferents maps, per a veure quins són més rellevants a l'hora de 'enganxar' amb el receptor.  %XXX enganxar

Es tracta d'un algoritme evolutiu relativament senzill pel que fa al propi algoritme, però més
complex pel referent a les condicions externes que s'imposen per la natura, i que ens obliguen a
posar condicions i limitacions al Algoritme Evolutiu (als Algoritmes Evolutius no els agraden les 
limitacions )

Com es mostrarà en les següents seccions, hi ha una part important de feina
realitzada en la preparació de les dades, i en la corresponent validació.
% section Objectius (end)

\section{Context Químic} % (fold)
\label{sec:Context Quimic}
Tanimotos, conformacions, y tal 
% section Context Químic (end)

\section{Procediment informàtic} % (fold)
\label{sec:Procediment informatic}


% section inf (end)
\section{preparació de dades} % (fold)
\label{sec:preparacio de dades}
Una molècula es pot trobar en la natura en diverses formes.  Les clasificacions es divideixen en
isomers i conformacions.  Una molècula pot tenir diversos isomers, i cada isomer es pot trobar en  
diverses conformacions.

És a dir, que per un nom (o id), en la natura deriven diverses estructures, que comparteixen nom, i
elements, però poden estar organitzats de forma diferent en l'espai. Aquestes organitzacions son
simètriques en l'espai 3D, donant lloc a molècules similars, però que en l'espai 3D, no es superposen de la
mateixa forma. (exemple, les mans, que no es poden superposar en 3D).

Les conformacions, en canvi, són variacions de la posició dels àtoms en l'espai, però mantenint els
enllaços entre àtoms.

Per passar d'un estereoisomer a un altre, s'han de trencar enllaços, i tornar-se a formar.  En les
conformacions no passa això, i es pot passar d'una a una altra, només canviant els angles dels
enllaços i la seva flexibilitat. 

Donat un conjunt de molècules que volem utilitzar, hem de saber quina de les conformacions triarem
per a cada molècula.  Per a fer la \texttt{explosió} de combinacions, s'utilitza un sofware
especific, que fa la combinatoria i genera les possibles variacions sobre una molècula donada. 

\subsection{Procés de preparació} % (fold)
\label{sub:Proces de preparacio}

Amb un conjunt de molècules, l'hi hem de trobar tots els isomers i les conformacions diferents en
les que es poden trobar a la natura, és per això que primer, donada una base de dades de molècules
diferents, hem de generar totes les conformacions possibles de cada molècula.  Una vegada les tenim
generades, hem de quedar-nos amb la conformació que millors resultats dóna en aquest problema.  Per
fer això, tenim un seguit de programes de tractament de dades, que ens permeten prepararles d'acord
amb les nostres necessitats.
% subsection Procés de preparació (end)

\subsection{Massatgeant les dades} % (fold)
\label{sub:Massatgeant les dades}

% subsection Massatgeant les dades (end)

% section preparació de dades (end)

\section{Algoritme genètic} % (fold)
\label{sec:Algoritme genetic}

\subsection{Fitness} % (fold)
\label{sub:Fitness}
En aquest problema, la funció que ens indica com de bò és un individu (funció \texttt{fitness}) és
el resultat de evaluar la ordenació que dóna la ponderació indicada per l'individu, respecte els
nostres coneixements de activitat. Aquesta evaluació la fem mitjançant un dels diferents algoritmes
de evaluació d'ordenacions.

\begin{itemize}
	\item Roc
	\item BedRoc
	\item Enrichment Factor
	\item \dots
\end{itemize}


En qualsevol d'aquests procediments, es tracta d'evaluar com està de ben ordenada una llista, en
funció d'unes regles.

% subsection Fitness (end)
\subsection{Operadors} % (fold)
\label{sub:Operadors}
% subsection Operadors (end)

Els operadors usats en aquest algoritme genètic són els més clàssics ja que donat el problema, tan
sols hi ha necessitat d'aplicar operadors ``bàsics'' als individus.  

\subsubsection{Crossover} % (fold)
\label{ssub:Crossover}
Pel creuament s'ha provat un creuament per un punt, i el creuament per dos punts, donant millors
resultats el creuament en tant sols un punt.  Amb una provabilitat d'un 0.8 es fa creuament i el
``fill'' és tan sols una partició formada a partir dels 2 pares, amb un punt de tall aleatori.
% subsubsection crossover (end)

\subsubsection{Mutacions} % (fold)
\label{ssub:Mutacions}

S'apliquen mutacions en un 0.1\% dels individus una vegada fet el creuament.  Les mutacions en
aquesta aplicació consisteixen en modificar una ponderació per una altra (al·lel).  La única
restricció és que es mantingui dins dels marges [-1,+1] .
% subsubsection Mutacions (end)

\subsection{sobreentrenament} % (fold)
\label{sub:sobreentrenament}

Un dels problemes que comporta la utilització d'algoritmes evolutius és
l'anomenat sobreentrenament.  El sobreentrenament és l'efecte que es dóna quan
utilitzem un conjunt de dades per entrenar un programa, i l'utilitzem tantes
vegades, que la solució obtinguda és molt bona pel conjunt de dades amb el qual
ha estat entrenat, però al aplicar la solució obtinguda en altres dades, els
resultats obtinguts són molt pitjors als esperats.

Això es pot donar per diversos motius, però un dels motius més freqüents és quan
disposem de poques dades en el nostre conjunt d'entrenament.  L'algoritme
genètic, prepara una solució ``a mida'', donant-nos molt bons resultats en les
nostres dades, perquè ``estudia'' els casos del conjunt de entrenament un a un,
creant solucions ad-hoc, no aplicables a altres conjunts de dades.

Per tenir controlat aquest sobreentrenament, s'ha aplicat una tècnica basada en
``leave one out'' on fem l'entrenament amb un percentatge determinat de les
dades, 

- Pq no es sobreentreni, conjunt de test i conj de validació.
- 80 10 10, es fa sobre molecules i despres es deixen passar totes les
conformacions d'aquestes molecules, fent que les dades finals no siguin 80 10 10
, pero aixo es correcte pq la flexibilitat d'una molecula es part de la seva
'qualitat'.

% subsection sobreentrenament (end)

% section Al (end)

\end{document}
